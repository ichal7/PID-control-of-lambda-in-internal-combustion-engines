\chapter{Introduction}

\section{Motivation}
Internal Combustion Engines are one of the major causes of air pollution. By the late 1950s increased use of cars in major cities \cite{haagen1955automobile} \cite{haagen1953ozone} had resulted in serious concerns about air quality and human health. The three types of exhaust emissions are: hydrocarbons, oxides of nitrogen and carbon monoxide. A.J.Haägen-Smit showed that photochemical reactions among hydrocarbons and nitrogen oxides produce the many secondary pollutants that reduce visibility and cause eye and nose irritation in the Los Angeles area \cite{la}. Furthermore, later studies showed that the above exhaust gas pollutants are cancerous gases which can also cause heart diseases and respiratory disorders.

Fortunately, the amounts of these pollutants could be considerably decreased by use of a three way catalytic converter. For a three way catalyst to work properly, air-to-fuel ratio (lambda) should by close to its stoichiometric value \cite{Nielsen_2000}. As a result, closed-loop control of air-to-fuel ratio is of high importance.

\section{Research topic and Scope}

Regarding all the above, the aim of this Thesis is to construct a closed-loop control scheme to sufficiently regulate air-to-fuel resulting in a fully functional catalytic converter. The first part of this work involves obtaining a proper dynamic model of an SI engine and using it to setup a corresponding simulation in MATLAB. The second part involves building the closed loop scheme, by implementing and tuning a PID controller. 

\section{Outline}

Chapter \ref{ch:2} introduces some parts of the internal combustion engine required for understanding the need for controlling air-to-fuel ratio. These parts are the lambda sensor which determines the air-to-fuel ratio and the three way catalytic converter which converts gases and pollutants in the car exhaust into harmless compounds.

In Chapter \ref{ch:3}, control terminology and basic concepts are discussed. The objective is to understand the importance of feedback loop in almost every dynamic system. Afterwards, the concept of a controller is introduced for the purpose of highlighting the PID controller and its use.

Chapter \ref{ch:4}, thoroughly describes the case study of this Thesis. It starts with the dynamic model used for simulating an SI engine using the MATLAB software. This model is the mean value engine model (MVEM), which is described as the state of the art in the control oriented model of internal combustion engines. Thereafter, a closed-loop control is designed with a PID controller. 

In Chapter \ref{ch:5}, the performance of the closed-loop control is evaluated and the results from the simulations in Chapter \ref{ch:4} are discussed. 

In Chapter \ref{ch:6}, we reiterate and summarize on the findings from the previous chapters and highlight future work that can be done to further improve air-to-fuel ratio control.




