\chapter{Control}\label{ch:3}

Control theory and control engineering deal with dynamic systems such as internal combustion engines, aircrafts, industrial manufacturing, conveyor belts, reading and writing from hard drives, other processes such as keeping satellites in place above the earth, sun tracking control of solar collectors and last but not least several biological functions. 

The objective is to sufficiently control the aforementioned systems. This includes a system with good stability, robustness to exogenous disturbances and minimum oscillations. To achieve this, a controller and a feedback action are necessary.


\section{Feedback Loop}
Feedback is a process occurred when the output of a system is returned back to a point as an input. This process generates a chain of cause and effect called the \textit{feedback}.

Karl Astrom, one of the most prolific contributors to control theory, states that \textit{the magic of feedback is that it can create a system that performs well from components that perform poorly}
\cite{aastrom2006challenges}.

A feedback control system at its basic form has three components: a system that needs to be controlled which is called \textit{plant}, a controller to alter the input signal of the plant and a feedback loop.

\begin{figure}[H]
    \centering
    \includegraphics[width=1.0\textwidth]{feedbackloop}
    \caption{A schematic diagram of a general automatic control system}
    \label{fig:feedback}
\end{figure}

The elements of a closed loop system are illustrated in Figure \ref{fig:feedback}. Below, we describe these elements and the information, or signals \footnote{a signal is a function that conveys information about a phenomenon}, that flow between elements:

\begin{itemize}
\item The dependent variable, called \textbf{output} is  the quantity or signal of the controlled system that is directly measured and controlled.
\item The independent variable, called \textbf{input} or \textbf{desired value} is the value that the output variable needs to converge with.
\item The output \textbf{signal of the feedback} is a signal of the controlled output which is sent back to the input to be compared with the desired value.
\item The \textbf{error}  is the difference between the desired value and the controlled output.
\item The \textbf{disturbance} is any exogenous change that affects the output signal.
\end{itemize}

\subsection{Control Properties}

\subsubsection*{Stability}
A stable system is a system that produces a bounded
output for a given bounded input \cite{dorf2011modern}. Generally for a stable system oscillations must die out as early as possible or steady state\footnote{when the state variable is unchanging in time} should be reached fast.

Stability is typically the first property considered in constructing control systems since unstable systems are not often used.

In figure \ref{fig:stable_unstable} there are illustrated some examples of stable and unstable systems. 
If the output variable converges, the system is said to be stable. Otherwise, it is an unstable system or it is marginally stable.

\begin{figure}[H]
    \centering
    \includegraphics[width=1.0\textwidth]{stable_unstable}
    \caption{Stability of different systems \cite{stability}}
    \label{fig:stable_unstable}
\end{figure}

\subsubsection*{Accuracy}

The control system is accurate if the output signal converges (or reaches sufficiently close) to the desired value, or more generally if an actual controlled process approaches the desired process.

\subsubsection*{Robustness}
Most of the times, control systems are subjected to disturbances which are unwanted signals that alter the desired behaviour. With the use of a feedback loop the sensitivity of a control system to these disturbances can be reduced.

Robustness is the ability of a closed loop control system to function properly when subjected to exogenous disturbances. More specifically as \cite{grimble2006robust} reports robustness is the ability of a closed loop system to be able to sufficiently  ignore exogenous noise. 



\subsubsection*{Miscellaneous}

If a control system converges quickly to its desired state it is said to have short \textbf{settling time}. As defined by Tay, Mareels and Moore in \cite{tay1998high} settling time is the time required for the response curve to reach and stay within a range of certain percentage (usually 5\% or 2\%) of the final value.

In a control system \textbf{overshoot} occurs when the output signal exceeds the desired input. 
As defined by Katsuhiko Ogata in \cite{ogata1995discrete} overshoot is the maximum peak value of the response curve measured from the desired response of the system.


\section{PID Controller}

In general, the task of a controller is to maintain a desired system performance while coping with possible system disturbances. In the case of the PID controller, to achieve a desired performance, the user needs to select carefully the amount of each control action: proportional, integral, and derivative \cite{aastrom1995pid}.

The controller can either be used with all its parts or with only the P, PI or PD terms. The proportional part reacts to the present control errors, the integral part sums up previous control errors and the derivative part predicts future control errors by using the derivative of the control error.

 \begin{figure}[ht]
    \centering
    \includegraphics[width=0.5\textwidth]{pid}
    \caption{A block diagram of a PID controller \cite{wikixxx}}
    \label{fig:pidblock}
\end{figure}

The controller should be adjusted properly or else desired performance will not be achieved. For example if the three gains of a PID controller are not chosen correctly, oscillations and instability can occur.

Based on a survey of over eleven thousand controllers in the refining, chemicals and pulp and paper industries, 97 percent of regulatory controllers utilize PID feedback \cite{desborough2000regulatory}.
 
 \begin{figure}[ht]
    \centering
    \includegraphics[width=1.0\textwidth]{pastpresentfuture}
    \caption{A PID controller takes control action based on past, present and
prediction of future control errors \cite{PidC8}}
    \label{fig:stable_unstable}
\end{figure}

\subsection{P term}

The proportional term of a PID controller depends only on the difference between the desired input value and the output signal, which is also known as the error signal. Thus, a significant usage of the P controller is to decrease the steady state error of the system \cite{pid}.

It also defines the ratio of output signal to the error signal. As the proportional gain factor  increases, the steady state error of the system decreases. However, despite the reduction, P control can never manage to eliminate the steady state error of the system. 

As we increase the proportional gain, it provides smaller amplitude and phase margin. Also, raising the P term will boost  the speed of the control system response (shorter settling time). This controller is used by itself when the control system can withstand a constant steady state error.

It can be easily concluded that applying P controller decreases the rise time, but if the P term is too big, oscillations will occur to the output signal. Further increase of P gain will result in larger oscillations and the system may become unstable.


\subsubsection*{Transient Response}

 \textbf{Advantages} of using P controller to control a second order plant:

\begin{itemize}
\item Increasing gain decreases rise time.
\item Increasing gain decreases steady state error.
\end{itemize}
\textbf{Disadvantages} of using P controller to control a second order plant:

\begin{itemize}
\item Increasing gain increases percent overshoot and oscillations.
\item Steady state is never zero if only P controller is used.
\end{itemize}

\subsection{I term}

Due to limitation of the Proportional controller where the desired input value cannot be equal to the output variable, there is a need to eliminate the steady state error. With the help of Integral Controller zero steady state error\footnote{Steady-State error is the final difference between the output variable and the desired value} can be achieved \cite{pid}.

The Integral Controller integrates the error signal over a period of time. This behaviour will continue to go on. If the error becomes zero the output of the I term will remain constant.

\subsubsection*{Transient Response}

Integral action eliminates steady state error. However, it has very poor transient response. Using integral action increases the oscillations in the output of the closed loop systems, as well as it limits the speed of response and affects stability of the system.

\subsection{D term}

The aim of using Derivative controller is to increase the stability of the control system by controlling it more successfully thanks to the ability to predict the future error of the system response. Derivative control is the reaction to the rate of change of error according to time, multiplied by derivative constant. Because of the function of Derivative control sudden change in the value of the error signal must be avoided. In order to do so, the derivative is taken from the output signal of the state variable instead of the error signal \cite{pid}.

In conclusion, increasing the derivative gain parameter will cause the control system to react more sufficiently to changes in the error signal and will reduce oscillations. On the other hand, most practical control systems use very small derivative gain, because the higher the derivative action the less robust the system becomes to exogenous disturbances. If the feedback signal is oscillating, the derivative action can make the whole system unstable. Thus, it is important to state that derivative controller is never used without P controller.

\subsubsection*{Transient Response}

Derivative controller is usually used to improve transient response of the closed loop system. Derivative term decreases oscillations, but it increases high frequency noise.

\subsection{PID Tuning}

Tuning a control loop is setting the optional control gains for P, I and D to their optimum values in order to get an ideal response from a control system.

The first requirement that should be met is the stability of the control system. Beyond that, different systems have different requirement and different behaviours are expected.

Although the PID controller has only three terns, it is not easy to find good values for them, because the criteria might not be compatible with each other. 
  
PID tuning is mostly a heuristic concept but the existence of many objectives to be met such as no oscillation, good stability and low rise time and settling time makes this process harder. For example sometimes, systems might have nonlinearity problem which means that while the parameters works properly for full load conditions, they might not work as effective for no load conditions. Also, if the PID parameters are not well tuned, the state variable may become unstable causing oscillations that result in mechanical failure.

For a system to work properly, the output signal should be stable, as well as it should not oscillate. However, there are case where some oscillations are acceptable.

\subsubsection*{Trial and Error Method}
One of the many methods for tuning a PID controller is the trial and error method. To tune a PID controller using this method one should completely understand the action of each term.

In this method, the Integral and Derivative terms are set to zero first, and the proportional gain is increased until the system starts oscillating. As the proportional gain is increased, rise time and settling time are decreasing, until a threshold where the system becomes unstable. At the time proportional term is tuned to make the system faster and the closed loop system oscillates at a constant rate,  the integral term is adjusted so that the oscillations will be gradually reduced. The integral controller minimizes the steady state error, but at the same time boosts overshoot. At this point in time proportional and derivative gains have been tuned. The parameter remained to be tuned is the derivative one. Derivative gain is increased  until the system reaches quickly its desired value. Increasing derivative term decreases the overshoot of the output signal.
 

 		