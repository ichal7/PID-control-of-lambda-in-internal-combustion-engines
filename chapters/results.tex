\chapter{Results and Discussion}\label{ch:5}
In chapter 4 we described the mean value engine model for simulating an SI engine using MATLAB software. We proceed to connect the MVEM with a PID controller and finally tune the controller using the trial and error method. 

In this chapter we will first present and discuss the results of controlling lambda with a PID controller when using a constant value for the throttle plate angle. Then the results when treating the throttle plate angle as a disturbance variable will be presented and discussed.

\section{Constant throttle plate angle}
For this simulation the following considerations are made:
\begin{itemize}
	\item Throttle plate angle $\alpha$ is considered as constant and equal to 25 degrees.
	\item The single input $\dot{m_{fi}}$ is the output signal of the PID controller. 
	\item The history of the MVEM is considered as presented in section 3.2.1.3 
	\item Desired input value is set to 1.
	\item Simulation time is set to 10 seconds.
\end{itemize} 
As one can see from Figure \ref{fig:a25} the results are satisfactory. First and foremost the steady state error is zero. Furthermore, the system response is very quick. The lambda value stabilizes at around 0.7 seconds which is much less than the time that the engine needs to warm up. One should also note that overshoot is not remarkable, it reaches up to 1.155 which is very close to the stoichiometric value of lambda. Last but not least, oscillations are almost zero, which means that the system response is really adequate.

\begin{figure}[H]
    \centering
    \includegraphics[width=1.0\textwidth]{a25}
    \caption{Lambda signal for constant throttle plate angle}
    \label{fig:a25}
\end{figure}


\section{Disturbance throttle plate angle}

For this simulation the following considerations are made:

\begin{itemize}
	\item Throttle plate angle $\alpha$ is considered as noise disturbance .
	\item The  input $\dot{m_{fi}}$ is the output signal of the PID controller. 
	\item The history of the MVEM is considered as presented in section 3.2.1.3 
	\item Desired input value is set to 1.
	\item Simulation time is set to 20 seconds.
\end{itemize} 
As one can see from Figure \ref{fig:adisturbance} the results are very different from the previous case. First and foremost, in this case the lambda signal does not converge to the desired value which is equivalent to one. Another noteworthy statement is that in this case overshoot reaches up to 1.8 which is a value that is quite far from the stoichiometric one, which is one. Last but not least, one can observe that oscillations are quite remarkable and as a result affect the stability of the system.

Even though the results on the first sight seem unsatisfactory, they are not. The fact that lambda does not converge is only due to the disturbance which changes at each time step. Furthermore, one should take into consideration the simulated model on the SI engine. The mean value engine model is a nonlinear four state dynamic model and as one knows PID controllers cannot effectively control nonlinear models.
Last but not least, studies have shown that classic controllers, such as PI systems, could not result in robust control of lambda against disturbances \cite{lambdacontrol}. A nonlinear controller could perform better, but it wouldn't converge either in the presence of disturbances. 
\\

\begin{figure}[ht!]
    \centering
    \includegraphics[width=1.0\textwidth]{with_distrurbance_a}
    \caption{Lambda signal for throttle plate angle as disturbance}
    \label{fig:adisturbance}
\end{figure}

