\chapter{Conclusion and Future Work} \label{ch:6}

\section{Conclusion}
The main objective of this Diploma Thesis was to control lambda in internal combustion engines using a PID controller. Firstly, we discussed about some parts of internal combustion engines which are of high importance about minimizing exhaust pollutants and gas emissions. Furthermore, we explained the functionality and applications of feedback control and highlighted the importance of the PID controller. Last but not least, we conducted a case study about controlling lambda in internal combustion engines using a PID controller.

The case study was conducted completely in MATLAB software. The engine simulated in MATLAB was a 1275cc British Leyland engine which had been mapped in Laboratory for Energetics, Technical University of Denmark. The simulation model was the mean value engine model, which is a nonlinear four state dynamic model of an SI engine. The controller used was a PID controller which had been tuned using the trial and error method.

Finally, we illustrated the results from two simulations. In the first simulation the input of the MVEM was the injected fuel mass flow, while in the second simulation throttle plate angle was added as noise disturbance. The air-to-fuel ratio (lambda) diagrams against time were pretty different for the two simulations. More specifically, while on the first simulation the PID ability to control lambda is very satisfactory, when noise disturbance was added to the system, oscillations occurred and steady state error was different from zero.


\section{Future Work}
Results can get better for both simulations. The PID controller, could not result in robust control of lambda against exogenous disturbances and modelling uncertainties \cite{zhai2009neural}. If one wants to control lambda within 1 \% of its stoichiometric value, other control techniques should be applied.

Nowadays, microcomputers have become all the more powerful and they can be applied in a lot of different situations. Electronic Control Units (ECUs) of engines can implement intelligent and adaptive control algorithms of lambda. The performance of these controllers is improved compared to classic control schemes, such as the PID controller. These controllers are robust to exogenous disturbances and model uncertainties as well as, they have significantly better transient and steady state performances.

  
